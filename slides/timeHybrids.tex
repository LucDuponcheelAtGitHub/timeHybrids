\documentclass[11pt]{beamer}

\usepackage[utf8]{inputenc}
\usepackage{amssymb}
\usepackage{hyperref}

\usetheme{Singapore}

\definecolor{DarkRed}{rgb}{0.50,0,0}
\definecolor{DarkGreen}{rgb}{0,0.50,0}
\definecolor{DarkBlue}{rgb}{0,0,0.50}
\definecolor{Black}{rgb}{0,0,0}

\newcommand{\ttdr}[1]{{\tt{\color{DarkRed} #1}}}
\newcommand{\emdr}[1]{{\em{\color{DarkRed} #1}}}
\newcommand{\ttdg}[1]{{\tt{\color{DarkGreen} #1}}}
\newcommand{\emdg}[1]{{\em{\color{DarkGreen} #1}}}
\newcommand{\ttdb}[1]{{\tt{\color{DarkBlue} #1}}}
\newcommand{\emdb}[1]{{\em{\color{DarkBlue} #1}}}

\setbeamertemplate{frametitle}{
  \begin{centering}
    {\Large \textbf{\textmd{\insertframetitle}}}
  \end{centering}
}

\setbeamertemplate{navigation symbols}{}

\setbeamertemplate{footline}{
  \begin{center}
    {\color{DarkBlue}{\large
        \insertframenumber}
      \hspace{280pt}
      \includegraphics[height=20pt]{png/cycling.png}}
  \end{center}
}

\begin{document}

\begin{frame}
  \vspace{25pt}
  \begin{center}
    \LARGE{
      {\color{DarkGreen}{
            Time Hybrids \\}}
      \vspace{10pt}
      {\color{DarkRed}{
        A New Generic Theory of Reality\\}}
      \vspace{10pt}
      {\color{DarkBlue}{
          Fred Van Oystaeyen}}
    }
  \end{center}
\end{frame}

\begin{frame}
  \vspace{25pt}
  \begin{center}
    \LARGE{
      \vspace{10pt}
      {\color{DarkGreen}{
        Time Hybrids \\}}
      \vspace{10pt}
      {\color{DarkRed}{
          Trying to understand the book \\
          using Programmatic Notation \\}}
      \vspace{10pt}
      {\color{DarkBlue}{
          Luc Duponcheel \\}}
    }
  \end{center}
\end{frame}

\begin{frame}
  \vspace{25pt}
  \begin{center}
    \LARGE{
      \vspace{10pt}
      {\color{DarkRed}{
            WARNING 1 \\}}
      \vspace{10pt}
    }
    \large{
      My understanding of the book \\
      may not correspond with what Fred had in mind.
    }
  \end{center}
\end{frame}

\begin{frame}
  \vspace{25pt}
  \begin{center}
    \LARGE{
      \vspace{10pt}
      {\color{DarkRed}{
            WARNING 2 \\}}
      \vspace{10pt}
    }
    \large{
      Only a small part of the book will be dealt with.
    }
  \end{center}
\end{frame}

\begin{frame}
  \vspace{25pt}
  \begin{center}
    \LARGE{
      \vspace{10pt}
      {\color{DarkRed}{
            WARNING 3 \\}}
      \vspace{10pt}
    }
    \large{
      The natural language of this presentation is informal. \\
      \ldots \\
      but \\
      \ldots \\
      The programming language of this presentation is formal.
    }
  \end{center}
\end{frame}  

\begin{frame}
  \vspace{25pt}
  \begin{center}
    \LARGE{
      \vspace{10pt}
      {\color{DarkRed}{
            WARNING 4 \\}}
      \vspace{10pt}
    }
    \large{
      Do not expect me to go into all details. \\
      There is room for questions after the presentation.
    }
  \end{center}
\end{frame}


\begin{frame}
  \vspace{25pt}
  \begin{center}
    \LARGE{
      \vspace{10pt}
      {\color{DarkBlue}{
            Part One \\}}
      \vspace{10pt}
      Setting the scene.
    }
  \end{center}
\end{frame}

\begin{frame}[fragile]
  \frametitle{\begin{center}How the traditional theory models reality.\end{center}}
  \begin{itemize}
    \item<2-> Reality is modeled as being {\em continuous}.
    \item<3-> The {\em geometry} of the universe is modeled {\em analytically} \\
              using {\em differential geometry}.
    \item<4-> {\em Measuring} time moments and universe places play a fundamental role.
    \item<5-> Reality evolves from continuous intervals to discrete points \\
              driven by (measure based) shrinking time intervals. \\
  \end{itemize}  
\end{frame}

\begin{frame}[fragile]
  \frametitle{\begin{center}How the new theory models reality.\end{center}}
  \begin{itemize}
    \item<2-> Reality is modeled as being {\em discrete}.
    \item<3-> The geometry of the universe is modeled {\em algebraically} \\ 
              using {\em virtual topology and functor geometry}.
    \item<4-> {\em Ordering} time moments and universe places play a fundamental role.
    \item<5-> Reality evolves from discrete to continuous \\
              driven by (order based) growing time intervals. \\
  \end{itemize}  
\end{frame}

\begin{frame}[fragile]
  \frametitle{\begin{center}How the traditional theory models reality.\end{center}}
  \begin{itemize} 
    \item<2-> The geometry of the universe is {\em static}.
    \item<3-> {\em Speed} of {\em "something"} is defined {\em functorially} \\
              using a correspondence from function t -> t to function t -> s(t)
    \item<4-> Letting, in $\Delta$s/$\Delta$t, \\
              the {\em size} of the {\em moment interval} $\Delta$t and \\
              the {\em size} of the {\em place interval} $\Delta$s \\
              go to $0$.              
  \end{itemize}  
\end{frame}

\begin{frame}[fragile]
  \frametitle{\begin{center}How the traditional theory models reality.\end{center}}
  \begin{itemize}  
    \item<2-> Mathematically this approach from continuous intervals to discrete points is ok. \\
              But is it ok in reality?
    \item<3-> No, because there does not exist {\em "anything"} at any point. \\ 
              When moment interval sizes go to $0$, {\em nothing } is observable any more.
  \end{itemize}  
\end{frame} 

\begin{frame}[fragile]
  \frametitle{\begin{center}How the new theory models reality.\end{center}}
  \begin{itemize}  
    \item<2-> The geometry of the universe is {\em dynamic}~: \\
              {\em place transitions} are driven by {\em moment intervals}.
    \item<4-> {\em Movement} of {\em "something"} can still be defined {\em functorially}.
  \end{itemize}  
\end{frame}

\begin{frame}[fragile]
  \frametitle{\begin{center}How the new theory models reality.\end{center}}
  \begin{itemize}  
    \item<2-> Mathematically this approach from discrete to continuous is ok. \\
              But is it ok in reality?
    \item<3-> Yes, but reality will always be discrete.
  \end{itemize}  
\end{frame}  

\begin{frame}[fragile]
  \frametitle{\begin{center}Food for thought.\end{center}}
  \begin{itemize}  
    \item<2-> {\em "Something"} observed as continuous becomes discrete driven by shrinking places.
    \item<3-> {\em "Something"} being discrete becomes observed as continuous driven by growing time intervals.
  \end{itemize}  
\end{frame}  

\begin{frame}
  \vspace{25pt}
  \begin{center}
    \LARGE{
      \vspace{10pt}
      {\color{DarkBlue}{
            Part Two \\}}
      \vspace{10pt}
      Generic Theory
    }
  \end{center}
\end{frame}
  
\begin{frame}[fragile]
  \frametitle{\begin{center}Generic Theory\end{center}}
  \begin{itemize}
    \item<2-> I think of a {\em generic theory} \\
              as a \\
              {\em partially unifying specification theory} \\
              where \\
              {\em theories} are {\em specific implementations} of.
  \end{itemize}
\end{frame}
  
\begin{frame}[fragile]
  \frametitle{\begin{center}Theories of Reality\end{center}}
  \begin{itemize}
    \item<2-> {\em Relativity Theory}.
    \item<3-> {\em Quantum Theory}.
    \item<4-> Until now no {\em fully unifying theory} has been agreed upon.
  \end{itemize}
\end{frame}

\begin{frame}[fragile]
  \frametitle{\begin{center}Theories of Mathematics\end{center}}
  \begin{itemize}
    \item<2-> {\em Group Theory}.
    \item<3-> {\em Measure Theory}.
    \item<4-> \ldots
    \item<5-> I think of {\em Category Theory} \\
              as a {\em Generic Theory of Mathematics}.
  \end{itemize}
\end{frame}

\begin{frame}[fragile]
  \frametitle{\begin{center}Compositionality\end{center}}
  \begin{itemize}
    \item<2-> Compositionality is about {\em components}.
    \item<3-> Components can, \\
              starting from {\em atomic components}, \\
              be {\em composed} to {\em composite components}.
    \item<4-> Components come in two classes.
    \begin{itemize}
      \item<5-> {\em Funtion-like components}.
      \item<6-> {\em Data-like components}.
    \end{itemize}
    \item<7-> Both come with {\em limits}.
  \end{itemize}
\end{frame}

\begin{frame}[fragile]
  \frametitle{\begin{center}Correspondences (traditional)\end{center}}
  \begin{itemize}
    \item<2-> In the traditional theory \\
              there is a {\em "functorial" correspondence} between \\
              $t \rightarrow p(t)$, the {\em place function} of {\em "something"}, \\
              and \\
              $t \rightarrow v(t)$, the {\em velocity function} of {\em "something"}, \\
              as \\
              $v(t) = d(p(t))/dt$, a limit for time interval sizes going to $0$.
  \end{itemize}
\end{frame}

\begin{frame}[fragile]
  \frametitle{\begin{center}Correspondences (new)\end{center}}
  \begin{itemize}
    \item<2-> In the new theory \\
              {\em sets of "something"s} evolve to a limit for time interval sizes going to $\infty$. \\
              In the dynamic universe {\em places themselves} change driven by {\em time intervals}.
              {\em sets of "something"s} are mapped to places and \\
              the {\em movement} of {\em sets of "somethings"s} {\em after} resp {\em at} a time interval \\
              is defined as a {\em "functorial" correspondence} without any limits involved.
  \end{itemize}
\end{frame}

\begin{frame}[fragile]
  \frametitle{\begin{center}Category Theory\end{center}}
  \begin{itemize}
    \item<2-> A category consists of
      \begin{itemize}
      \item<3-> Data-like {\em objects}.
      \item<4-> Function-like {\em morphisms} between objects.
      \end{itemize}
    \item<5-> A category is function-like compositional.
    \begin{itemize}
      \item<6-> Morphisms can be {\em composed sequentially}.
    \end{itemize}
  \end{itemize}
\end{frame}

\begin{frame}[fragile]
  \frametitle{\begin{center}Category Theory\end{center}}
  \begin{itemize}
    \item<2-> A functor consists of
      \begin{itemize}
      \item<3-> A fuuncion that corresponds morphisms of one category to morphisms of another category.
      \end{itemize}
    \item<4-> A functor respects compositionality of morphisms
  \end{itemize}
\end{frame}

\begin{frame}
  \vspace{25pt}
  \begin{center}
    \LARGE{
      \vspace{10pt}
      {\color{DarkBlue}{
            Part Three \\}}
      \vspace{10pt}
      Specification and Implementations
    }
  \end{center}
\end{frame}

\begin{frame}[fragile]
  \frametitle{\begin{center}Specification\end{center}}
  \begin{itemize}
    \item<2-> {\em Declares} {\em features}.
    \item<3-> {\em Come with laws} for those declared features.
    \item<4-> Declared features, together with their laws, form the {\em requirements} of the specification.
    \item<5-> {\em Defines} features in terms of declared and defined features.
  \end{itemize}
\end{frame}
  
\begin{frame}[fragile]
  \frametitle{\begin{center}Implementations\end{center}}
  \begin{itemize}
    \item<2-> {\em Satisfy} the requirements of the specification. 
    \item<3-> {\em Define} declared features.
    \item<4-> {\em Come with proofs} of the laws for those defined features.
  \end{itemize}
\end{frame}

\begin{frame}
  \vspace{25pt}
  \begin{center}
    \LARGE{
      \vspace{10pt}
      {\color{DarkBlue}{
            Part Four \\}}
      \vspace{10pt}
      Writing a specification and its implementations programatically
    }
  \end{center}
\end{frame}

\begin{frame}[fragile]
  \frametitle{\begin{center}Composition\end{center}}
  {\color{DarkGreen}
    \begin{verbatim}
  trait Composition[Morphism[_, _]]:
  
    extension [Z, Y, X](ymx: Morphism[Y, X])
      def o(zmy: Morphism[Z, Y]): Morphism[Z, X]
      
    type Transition = [Z] =>> Morphism[Z, Z] 
    \end{verbatim}}
\end{frame}

\begin{frame}[fragile]
  \frametitle{\begin{center}Composition\end{center}}
  \begin{itemize}
    \item<2-> \ttdg{Morphism} \\
              is a {\em binary type constructor} parameter of \\
              \ttdg{Composition}.
    \item<3-> {\em Types} \ttdg{Z}, \ttdg{Y}, \ttdg{X}, \ldots {\em implicitly} model {\em (homogeneous) sets}.
    \item<4-> {\em Values} \ttdg{z}, \ttdg{y}, \ttdg{x}, \ldots {\em implicitly} model {\em elements}.
    \end{itemize}
\end{frame}

\begin{frame}[fragile]
  \frametitle{\begin{center}Composition\end{center}}
  {\color{DarkGreen}
    \begin{verbatim}
  trait Composition[Morphism[_, _]]:
  
    extension [Z, Y, X](ymx: Morphism[Y, X])
      def o(zmy: Morphism[Z, Y]): Morphism[Z, X]
      
    type Transition = [Z] =>> Morphism[Z, Z] 
    \end{verbatim}}
\end{frame}

\begin{frame}[fragile]
  \frametitle{\begin{center}Composition\end{center}}
  \begin{itemize}
    \item<2-> \ttdg{o} is a {\em sequential composition operator}.
    \item<3-> Values \ttdg{zmy} \ldots are morphisms from \ttdg{Z} to \ttdg{Y}.
    \end{itemize}
\end{frame}  

\begin{frame}[fragile]
  \frametitle{\begin{center}Composition\end{center}}
  {\color{DarkGreen}
    \begin{verbatim}
  trait Composition[Morphism[_, _]]:
  
    extension [Z, Y, X](ymx: Morphism[Y, X])
      def o(zmy: Morphism[Z, Y]): Morphism[Z, X]
      
    type Transition = [Z] =>> Morphism[Z, Z] 
    \end{verbatim}}
\end{frame}  

\begin{frame}[fragile]
  \frametitle{\begin{center}Composition\end{center}}
  \begin{itemize}
    \item<2-> \ttdg{Transition} involves only one type \ttdg{Z}.
    \end{itemize}
\end{frame}  

\begin{frame}[fragile]
  \frametitle{\begin{center}Composition\end{center}}
  {\color{DarkGreen}
    \begin{verbatim}
  trait Composition[Morphism[_, _]]:
  
    extension [Z, Y, X](ymx: Morphism[Y, X])
      def o(zmy: Morphism[Z, Y]): Morphism[Z, X]
      
    type Transition = [Z] =>> Morphism[Z, Z] 
    \end{verbatim}}
\end{frame}  

\begin{frame}[fragile]
  \frametitle{\begin{center}Associativity\end{center}}
  {\color{DarkGreen}
    \begin{verbatim}
  def associativity[Z, Y, X, W]
      : Morphism[Z, Y] => 
          Morphism[Y, X] =>
              Morphism[X, W] =>
              L[Morphism[Z, W]] =
    zmy =>
      ymx =>
        xmw =>
          {
              (xmw o ymx) o zmy
          } `=` {
              xmw o (ymx o zmy)
          }
    \end{verbatim}}
\end{frame}

\begin{frame}[fragile]
  \frametitle{\begin{center}Identity\end{center}}
  {\color{DarkGreen}
    \begin{verbatim}
  trait Identity[Morphism[_, _]]:
  
    def i[Z]: Morphism[Z, Z]
    \end{verbatim}}
\end{frame}

\begin{frame}[fragile]
  \frametitle{\begin{center}Category\end{center}}
  {\color{DarkGreen}
    \begin{verbatim}
  trait Category[Morphism[_, _]]
    extends Composition[Morphism],
        Identity[Morphism]:

    def composeAll[Z]
      : List[Transition[Z]] => Transition[Z] =
      _.foldLeft(i)(_ o _)
    \end{verbatim}}
\end{frame}

\begin{frame}[fragile]
  \frametitle{\begin{center}Functor\end{center}}
  {\color{DarkGreen}
    \begin{verbatim}
  trait Functor[
      FromMorphism[_, _]: Category,
      ToMorphism[_, _]: Category,
      Correspondence[_]
  ]:
      
    def f[Z, Y]: Function[
      FromMorphism[Z, Y],
      ToMorphism[Correspondence[Z], Correspondence[Y]]
    ]
    \end{verbatim}}
\end{frame}

\begin{frame}[fragile]
  \frametitle{\begin{center}Time, Universe, (Pre-)Things\end{center}}
  \begin{itemize}
  \item<2-> {\em We introduce a generic model for space-time where time is just a totally ordered set
            ordering the states of the universe at moments where over (not in) each state we define
            potentials or pre-things which are going to evolve via correspondences between the momentary
            potentials to existing things. Existing takes time. We can define a place function where some set of
            pre-things is mapped to a place of the topology of the universe}.
  \end{itemize}
\end{frame}

\begin{frame}[fragile]
  \frametitle{\begin{center}Time, Universe, (Pre-)Things\end{center}}
  \begin{itemize}
  \item<2-> {\em Pre-things} are artifacts of the {\em non-existing reality}.
  \item<3-> {\em Pre-things} are {\em momentary}.
  \item<4-> {\em Things} are artifacts of the {\em existing reality}.
  \item<5-> {\em Existing takes time.}
  \item<6-> {\em Observing takes even more time.}
  \item<7-> {\em What is it that we observe?}
  \end{itemize}
\end{frame}

\begin{frame}
  \vspace{25pt}
  \begin{center}
    \LARGE{
      \vspace{10pt}
      {\color{DarkBlue}{
            Part Five \\}}
      \vspace{10pt}
      Code fragments.
    }
  \end{center}
\end{frame}  

\begin{frame}[fragile]
  \frametitle{\begin{center}Time, Universe, (Pre-)Things\end{center}}
  \begin{itemize}
  \item<2-> fragment
  \item<3-> fragment 
  \item<4-> fragment
  \item<6-> fragment
  \item<7-> fragment
  \end{itemize}
\end{frame}

\begin{frame}[fragile]
  \frametitle{\begin{center}More Information\end{center}}
  \ttdg{https://github.com/LucDuponcheelAtGitHub/timeHybrids}
\end{frame}
  
\begin{frame}[fragile]
  \frametitle{\begin{center}THANKS FOR ATTENDING\end{center}}
  \begin{center}
    \includegraphics[height=190pt]{png/cycling.png}
  \end{center}
\end{frame}

\end{document}